%% Generated by Sphinx.
\def\sphinxdocclass{report}
\documentclass[letterpaper,10pt,english]{sphinxmanual}
\ifdefined\pdfpxdimen
   \let\sphinxpxdimen\pdfpxdimen\else\newdimen\sphinxpxdimen
\fi \sphinxpxdimen=.75bp\relax

\PassOptionsToPackage{warn}{textcomp}
\usepackage[utf8]{inputenc}
\ifdefined\DeclareUnicodeCharacter
% support both utf8 and utf8x syntaxes
  \ifdefined\DeclareUnicodeCharacterAsOptional
    \def\sphinxDUC#1{\DeclareUnicodeCharacter{"#1}}
  \else
    \let\sphinxDUC\DeclareUnicodeCharacter
  \fi
  \sphinxDUC{00A0}{\nobreakspace}
  \sphinxDUC{2500}{\sphinxunichar{2500}}
  \sphinxDUC{2502}{\sphinxunichar{2502}}
  \sphinxDUC{2514}{\sphinxunichar{2514}}
  \sphinxDUC{251C}{\sphinxunichar{251C}}
  \sphinxDUC{2572}{\textbackslash}
\fi
\usepackage{cmap}
\usepackage[T1]{fontenc}
\usepackage{amsmath,amssymb,amstext}
\usepackage{babel}



\usepackage{times}
\expandafter\ifx\csname T@LGR\endcsname\relax
\else
% LGR was declared as font encoding
  \substitutefont{LGR}{\rmdefault}{cmr}
  \substitutefont{LGR}{\sfdefault}{cmss}
  \substitutefont{LGR}{\ttdefault}{cmtt}
\fi
\expandafter\ifx\csname T@X2\endcsname\relax
  \expandafter\ifx\csname T@T2A\endcsname\relax
  \else
  % T2A was declared as font encoding
    \substitutefont{T2A}{\rmdefault}{cmr}
    \substitutefont{T2A}{\sfdefault}{cmss}
    \substitutefont{T2A}{\ttdefault}{cmtt}
  \fi
\else
% X2 was declared as font encoding
  \substitutefont{X2}{\rmdefault}{cmr}
  \substitutefont{X2}{\sfdefault}{cmss}
  \substitutefont{X2}{\ttdefault}{cmtt}
\fi


\usepackage[Bjarne]{fncychap}
\usepackage{sphinx}

\fvset{fontsize=\small}
\usepackage{geometry}


% Include hyperref last.
\usepackage{hyperref}
% Fix anchor placement for figures with captions.
\usepackage{hypcap}% it must be loaded after hyperref.
% Set up styles of URL: it should be placed after hyperref.
\urlstyle{same}

\usepackage{sphinxmessages}
\setcounter{tocdepth}{1}



\title{phlab}
\date{May 20, 2020}
\release{0.0.0.dev4}
\author{Andrey Geondzhian, Keith Gilmore}
\newcommand{\sphinxlogo}{\vbox{}}
\renewcommand{\releasename}{Release}
\makeindex
\begin{document}

\pagestyle{empty}
\sphinxmaketitle
\pagestyle{plain}
\sphinxtableofcontents
\pagestyle{normal}
\phantomsection\label{\detokenize{index::doc}}



\chapter{About phlab}
\label{\detokenize{about:about-phlab}}\label{\detokenize{about::doc}}
This package aims to create a convenient set of tools for fitting experimental data and obtaining electron\sphinxhyphen{}phonon coupling values from phonon contribution in resonant inelastic X\sphinxhyphen{}ray scattering cross\sphinxhyphen{}section.

It currently includes:
\begin{itemize}
\item {} 
Model for 1D harmonic oscillator interacting with a single electronic level.

\item {} 
Model for 2D harmonic oscillator (two modes active).

\item {} 
Model for 1D harmonic oscillator, distorted and displaced in the excited state (excited\sphinxhyphen{}state potential energy surface (PES), differs from the ground state one).

\end{itemize}


\chapter{Installation}
\label{\detokenize{installation:installation}}\label{\detokenize{installation::doc}}
It is a python based package and to install it you can simply run in terminal:

\sphinxcode{\sphinxupquote{\$pip install phlab}}

Download examples folder from this page and use it as a template for your projects. You may want to use Jupyter Notebooks, which comes with the examples. If you don’t have \sphinxhref{https://jupyter.org/documentation}{Jupyter} install it  via \sphinxcode{\sphinxupquote{pip}} as well:

\sphinxcode{\sphinxupquote{\$pip install jupyterlab}}


\chapter{Quick start}
\label{\detokenize{quickstart/index:quick-start}}\label{\detokenize{quickstart/index::doc}}
First things first

\begin{sphinxVerbatim}[commandchars=\\\{\}]
\PYG{k+kn}{import} \PYG{n+nn}{phlab}
\end{sphinxVerbatim}

Now let’s create our work space which is a wrapper for all the experiments and models:

\begin{sphinxVerbatim}[commandchars=\\\{\}]
\PYG{n}{workspace} \PYG{o}{=} \PYG{n}{phlab}\PYG{o}{.}\PYG{n}{rixs}\PYG{p}{(}\PYG{p}{)}
\end{sphinxVerbatim}

One of the main objects is a model. You can create any number of models and fit them to the exeperiment.
Here were are starting with single harmonic oscillator model. Check \sphinxcode{\sphinxupquote{./model\_name/}} for input and output files.

\begin{sphinxVerbatim}[commandchars=\\\{\}]
\PYG{n}{model} \PYG{o}{=} \PYG{n}{workspace}\PYG{o}{.}\PYG{n}{model\PYGZus{}single\PYGZus{}osc}\PYG{p}{(}\PYG{n}{name} \PYG{o}{=} \PYG{l+s+s1}{\PYGZsq{}}\PYG{l+s+s1}{1d}\PYG{l+s+s1}{\PYGZsq{}}\PYG{p}{)}
\end{sphinxVerbatim}

\begin{sphinxVerbatim}[commandchars=\\\{\}]
\PYG{n}{creating} \PYG{n}{model} \PYG{p}{:} \PYG{o}{/}\PYG{n}{Users}\PYG{o}{/}\PYG{n}{lusigeondzian}\PYG{o}{/}\PYG{n}{github}\PYG{o}{/}\PYG{n}{phlab}\PYG{o}{/}\PYG{n}{examples}\PYG{o}{/}\PYG{l+m+mi}{01}\PYG{n}{\PYGZus{}example}\PYG{o}{/}\PYG{l+m+mi}{1}\PYG{n}{d}
\PYG{o}{/}\PYG{n}{Users}\PYG{o}{/}\PYG{n}{lusigeondzian}\PYG{o}{/}\PYG{n}{github}\PYG{o}{/}\PYG{n}{phlab}\PYG{o}{/}\PYG{n}{examples}\PYG{o}{/}\PYG{l+m+mi}{01}\PYG{n}{\PYGZus{}example}\PYG{o}{/}\PYG{l+m+mi}{1}\PYG{n}{d}\PYG{o}{/}\PYG{n}{\PYGZus{}input}\PYG{o}{/}
\PYG{n}{no} \PYG{n+nb}{input} \PYG{n}{found}
\PYG{n}{creating} \PYG{n}{new} \PYG{n+nb}{input}
\PYG{n}{warning} \PYG{p}{:} \PYG{n}{please} \PYG{n}{check} \PYG{n}{new} \PYG{n+nb}{input}
\PYG{n}{number} \PYG{n}{of} \PYG{n}{models} \PYG{p}{:} \PYG{l+m+mi}{1}
\end{sphinxVerbatim}

Input is normally reading from \sphinxcode{\sphinxupquote{./model\_name/\_inputs/input\_model\_\{nm\}.json}} and is an atribute of the model

\begin{sphinxVerbatim}[commandchars=\\\{\}]
\PYG{n}{model}\PYG{o}{.}\PYG{n}{input}
\end{sphinxVerbatim}

\begin{sphinxVerbatim}[commandchars=\\\{\}]
\PYG{p}{\PYGZob{}}\PYG{l+s+s1}{\PYGZsq{}}\PYG{l+s+s1}{problem\PYGZus{}type}\PYG{l+s+s1}{\PYGZsq{}}\PYG{p}{:} \PYG{l+s+s1}{\PYGZsq{}}\PYG{l+s+s1}{rixs}\PYG{l+s+s1}{\PYGZsq{}}\PYG{p}{,}
 \PYG{l+s+s1}{\PYGZsq{}}\PYG{l+s+s1}{model}\PYG{l+s+s1}{\PYGZsq{}}\PYG{p}{:} \PYG{l+s+s1}{\PYGZsq{}}\PYG{l+s+s1}{1d}\PYG{l+s+s1}{\PYGZsq{}}\PYG{p}{,}
 \PYG{l+s+s1}{\PYGZsq{}}\PYG{l+s+s1}{method}\PYG{l+s+s1}{\PYGZsq{}}\PYG{p}{:} \PYG{l+s+s1}{\PYGZsq{}}\PYG{l+s+s1}{fc}\PYG{l+s+s1}{\PYGZsq{}}\PYG{p}{,}
 \PYG{l+s+s1}{\PYGZsq{}}\PYG{l+s+s1}{vib\PYGZus{}space}\PYG{l+s+s1}{\PYGZsq{}}\PYG{p}{:} \PYG{l+m+mi}{1}\PYG{p}{,}
 \PYG{l+s+s1}{\PYGZsq{}}\PYG{l+s+s1}{coupling}\PYG{l+s+s1}{\PYGZsq{}}\PYG{p}{:} \PYG{l+m+mf}{0.1}\PYG{p}{,}
 \PYG{l+s+s1}{\PYGZsq{}}\PYG{l+s+s1}{omega\PYGZus{}ph}\PYG{l+s+s1}{\PYGZsq{}}\PYG{p}{:} \PYG{l+m+mf}{0.195}\PYG{p}{,}
 \PYG{l+s+s1}{\PYGZsq{}}\PYG{l+s+s1}{nf}\PYG{l+s+s1}{\PYGZsq{}}\PYG{p}{:} \PYG{l+m+mf}{10.0}\PYG{p}{,}
 \PYG{l+s+s1}{\PYGZsq{}}\PYG{l+s+s1}{nm}\PYG{l+s+s1}{\PYGZsq{}}\PYG{p}{:} \PYG{l+m+mf}{100.0}\PYG{p}{,}
 \PYG{l+s+s1}{\PYGZsq{}}\PYG{l+s+s1}{energy\PYGZus{}ex}\PYG{l+s+s1}{\PYGZsq{}}\PYG{p}{:} \PYG{l+m+mf}{10.0}\PYG{p}{,}
 \PYG{l+s+s1}{\PYGZsq{}}\PYG{l+s+s1}{omega\PYGZus{}in}\PYG{l+s+s1}{\PYGZsq{}}\PYG{p}{:} \PYG{l+m+mf}{10.0}\PYG{p}{,}
 \PYG{l+s+s1}{\PYGZsq{}}\PYG{l+s+s1}{gamma}\PYG{l+s+s1}{\PYGZsq{}}\PYG{p}{:} \PYG{l+m+mf}{0.105}\PYG{p}{,}
 \PYG{l+s+s1}{\PYGZsq{}}\PYG{l+s+s1}{gamma\PYGZus{}ph}\PYG{l+s+s1}{\PYGZsq{}}\PYG{p}{:} \PYG{l+m+mf}{0.05}\PYG{p}{,}
 \PYG{l+s+s1}{\PYGZsq{}}\PYG{l+s+s1}{alpha\PYGZus{}exp}\PYG{l+s+s1}{\PYGZsq{}}\PYG{p}{:} \PYG{l+m+mf}{0.01}\PYG{p}{\PYGZcb{}}
\end{sphinxVerbatim}

If you wish to alter input inside your code just call the prameter your want to overwrite. (Note: input file will be overwritten when you will run the model \sphinxcode{\sphinxupquote{model.run()}})

\begin{sphinxVerbatim}[commandchars=\\\{\}]
\PYG{n}{model}\PYG{o}{.}\PYG{n}{input}\PYG{p}{[}\PYG{l+s+s1}{\PYGZsq{}}\PYG{l+s+s1}{coupling}\PYG{l+s+s1}{\PYGZsq{}}\PYG{p}{]} \PYG{o}{=} \PYG{l+m+mf}{0.15}
\PYG{n}{model}\PYG{o}{.}\PYG{n}{color} \PYG{o}{=} \PYG{l+s+s1}{\PYGZsq{}}\PYG{l+s+s1}{r}\PYG{l+s+s1}{\PYGZsq{}}
\end{sphinxVerbatim}

\begin{sphinxVerbatim}[commandchars=\\\{\}]
\PYG{n}{model}\PYG{o}{.}\PYG{n}{run}\PYG{p}{(}\PYG{p}{)}
\end{sphinxVerbatim}

Now let’s create the experiment. At the very least you have to specify the  path to file  with exp data.

\begin{sphinxVerbatim}[commandchars=\\\{\}]
\PYG{n}{exp} \PYG{o}{=} \PYG{n}{workspace}\PYG{o}{.}\PYG{n}{experiment}\PYG{p}{(}\PYG{n}{file} \PYG{o}{=} \PYG{l+s+s1}{\PYGZsq{}}\PYG{l+s+s1}{test\PYGZus{}data.csv}\PYG{l+s+s1}{\PYGZsq{}}\PYG{p}{,} \PYG{n}{name}\PYG{o}{=} \PYG{l+s+s1}{\PYGZsq{}}\PYG{l+s+s1}{ test}\PYG{l+s+s1}{\PYGZsq{}}\PYG{p}{)}
\end{sphinxVerbatim}

Now to visulize everything you can create vitem and list the models and experiment objects that you would like to compare.

\begin{sphinxVerbatim}[commandchars=\\\{\}]
\PYG{n}{vitem} \PYG{o}{=} \PYG{n}{workspace}\PYG{o}{.}\PYG{n}{visual}\PYG{p}{(}\PYG{n}{model\PYGZus{}list} \PYG{o}{=} \PYG{p}{[}\PYG{n}{model}\PYG{p}{]}\PYG{p}{,} \PYG{n}{exp} \PYG{o}{=} \PYG{n}{exp}\PYG{p}{)}
\end{sphinxVerbatim}

Note: \sphinxcode{\sphinxupquote{scale = 0}}  would normalizes everything by the maximum of intensity.

\begin{sphinxVerbatim}[commandchars=\\\{\}]
\PYG{n}{vitem}\PYG{o}{.}\PYG{n}{show}\PYG{p}{(}\PYG{n}{scale} \PYG{o}{=} \PYG{l+m+mi}{0}\PYG{p}{)}
\end{sphinxVerbatim}

\sphinxincludegraphics{{output_14_0}.png}


\chapter{Input description}
\label{\detokenize{input:input-description}}\label{\detokenize{input::doc}}

\section{Basic (1D oscillator)}
\label{\detokenize{input:basic-1d-oscillator}}



\section{2D oscillator (2 modes problem)}
\label{\detokenize{input:d-oscillator-2-modes-problem}}



\section{Displaced and Distorted Oscillator}
\label{\detokenize{input:displaced-and-distorted-oscillator}}



\chapter{Modules}
\label{\detokenize{modules/index:modules}}\label{\detokenize{modules/index::doc}}

\section{main}
\label{\detokenize{modules/main:module-phlab}}\label{\detokenize{modules/main:main}}\label{\detokenize{modules/main::doc}}\index{module@\spxentry{module}!phlab@\spxentry{phlab}}\index{phlab@\spxentry{phlab}!module@\spxentry{module}}\index{rixs (class in phlab)@\spxentry{rixs}\spxextra{class in phlab}}

\begin{fulllineitems}
\phantomsection\label{\detokenize{modules/main:phlab.rixs}}\pysiglinewithargsret{\sphinxbfcode{\sphinxupquote{class }}\sphinxcode{\sphinxupquote{phlab.}}\sphinxbfcode{\sphinxupquote{rixs}}}{\emph{\DUrole{n}{project\_name}\DUrole{o}{=}\DUrole{default_value}{\textquotesingle{}\textquotesingle{}}}, \emph{\DUrole{n}{out\_dir}\DUrole{o}{=}\DUrole{default_value}{\textquotesingle{}/\_output/\textquotesingle{}}}, \emph{\DUrole{n}{inp\_dir}\DUrole{o}{=}\DUrole{default_value}{\textquotesingle{}/\_input/\textquotesingle{}}}}{}
Class rixs exists as a wrapper around    both models and experiment objects.
\begin{description}
\item[{Args:}] \leavevmode\begin{description}
\item[{problem\_name:  str}] \leavevmode
name of the project.

\item[{out\_dir: str}] \leavevmode
name of the ouptu directory.

\item[{inp\_dir: str}] \leavevmode
name of the ouptu directory.

\end{description}

\item[{Attributes:}] \leavevmode\begin{description}
\item[{nmodel: int}] \leavevmode
number of models created within this project.

\item[{nexp: int}] \leavevmode
number of exp created within this project.

\item[{abs\_path: str}] \leavevmode
absolute path to the working directory

\end{description}

\end{description}
\index{experiment() (phlab.rixs method)@\spxentry{experiment()}\spxextra{phlab.rixs method}}

\begin{fulllineitems}
\phantomsection\label{\detokenize{modules/main:phlab.rixs.experiment}}\pysiglinewithargsret{\sphinxbfcode{\sphinxupquote{experiment}}}{\emph{\DUrole{n}{file}\DUrole{o}{=}\DUrole{default_value}{\textquotesingle{}\textquotesingle{}}}, \emph{\DUrole{n}{col}\DUrole{o}{=}\DUrole{default_value}{{[}0, 1{]}}}, \emph{\DUrole{n}{name}\DUrole{o}{=}\DUrole{default_value}{\textquotesingle{}\textquotesingle{}}}}{}
Experiment.
\begin{description}
\item[{Args:}] \leavevmode\begin{description}
\item[{file: str}] \leavevmode
path to the file with the exp data

\item[{col: list}] \leavevmode
{[}column x ; column y{]} defines which columns to read from the file

\item[{name: str}] \leavevmode
name of the experiment

\end{description}

\item[{Returns:}] \leavevmode\begin{description}
\item[{experiment.experiment(): object}] \leavevmode
calls experiment sub\sphinxhyphen{}package

\end{description}

\end{description}

\end{fulllineitems}

\index{model\_dist\_disp\_osc() (phlab.rixs method)@\spxentry{model\_dist\_disp\_osc()}\spxextra{phlab.rixs method}}

\begin{fulllineitems}
\phantomsection\label{\detokenize{modules/main:phlab.rixs.model_dist_disp_osc}}\pysiglinewithargsret{\sphinxbfcode{\sphinxupquote{model\_dist\_disp\_osc}}}{\emph{\DUrole{n}{name}\DUrole{o}{=}\DUrole{default_value}{\textquotesingle{}\textquotesingle{}}}}{}
Model describing distorted and displaced in the excited\sphinxhyphen{}state harmonic oscillator
which interacts with a single electronic level.
\begin{description}
\item[{Args:}] \leavevmode\begin{description}
\item[{name: str}] \leavevmode
name of the model

\end{description}

\item[{Note:}] \leavevmode
input and output files are located inside  ‘./name/’ directory

\item[{Returns:}] \leavevmode\begin{description}
\item[{model.dist\_disp\_osc(): object}] \leavevmode
calls model sub\sphinxhyphen{}package

\end{description}

\end{description}

\end{fulllineitems}

\index{model\_double\_osc() (phlab.rixs method)@\spxentry{model\_double\_osc()}\spxextra{phlab.rixs method}}

\begin{fulllineitems}
\phantomsection\label{\detokenize{modules/main:phlab.rixs.model_double_osc}}\pysiglinewithargsret{\sphinxbfcode{\sphinxupquote{model\_double\_osc}}}{\emph{\DUrole{n}{name}\DUrole{o}{=}\DUrole{default_value}{\textquotesingle{}\textquotesingle{}}}}{}
Model describing 2D harmonic oscillator which interacts
with a single electronic level.
\begin{description}
\item[{Args:}] \leavevmode\begin{description}
\item[{name: str}] \leavevmode
name of the model

\end{description}

\item[{Note:}] \leavevmode
input and output files are located inside  ‘./name/’ directory

\item[{Returns:}] \leavevmode\begin{description}
\item[{model.double\_osc(): object}] \leavevmode
calls model sub\sphinxhyphen{}package

\end{description}

\end{description}

\end{fulllineitems}

\index{model\_single\_osc() (phlab.rixs method)@\spxentry{model\_single\_osc()}\spxextra{phlab.rixs method}}

\begin{fulllineitems}
\phantomsection\label{\detokenize{modules/main:phlab.rixs.model_single_osc}}\pysiglinewithargsret{\sphinxbfcode{\sphinxupquote{model\_single\_osc}}}{\emph{\DUrole{n}{name}\DUrole{o}{=}\DUrole{default_value}{\textquotesingle{}\textquotesingle{}}}}{}
Model describing a harmonic oscillator interacting
with a single electronic level.
\begin{description}
\item[{Args:}] \leavevmode\begin{description}
\item[{name: str}] \leavevmode
name of the model

\end{description}

\item[{Note:}] \leavevmode
input and output files are located inside  ‘./name/’ directory

\item[{Returns:}] \leavevmode\begin{description}
\item[{model.single\_osc(): object}] \leavevmode
calls model sub\sphinxhyphen{}package

\end{description}

\end{description}

\end{fulllineitems}

\index{visual() (phlab.rixs method)@\spxentry{visual()}\spxextra{phlab.rixs method}}

\begin{fulllineitems}
\phantomsection\label{\detokenize{modules/main:phlab.rixs.visual}}\pysiglinewithargsret{\sphinxbfcode{\sphinxupquote{visual}}}{\emph{\DUrole{n}{model\_list}\DUrole{o}{=}\DUrole{default_value}{{[}{]}}}, \emph{\DUrole{n}{exp}\DUrole{o}{=}\DUrole{default_value}{{[}{]}}}}{}
Creates visual object within the current project (works space).
\begin{description}
\item[{Args:}] \leavevmode\begin{description}
\item[{model\_list:  list}] \leavevmode
list of models to plot

\item[{exp: object}] \leavevmode
experiment to plot

\end{description}

\item[{Returns:}] \leavevmode\begin{description}
\item[{visual.plot():  object}] \leavevmode
calls visual sub\sphinxhyphen{}package

\end{description}

\end{description}

\end{fulllineitems}


\end{fulllineitems}



\section{models}
\label{\detokenize{modules/model:module-phlab.model}}\label{\detokenize{modules/model:models}}\label{\detokenize{modules/model::doc}}\index{module@\spxentry{module}!phlab.model@\spxentry{phlab.model}}\index{phlab.model@\spxentry{phlab.model}!module@\spxentry{module}}\index{dist\_disp\_osc (class in phlab.model)@\spxentry{dist\_disp\_osc}\spxextra{class in phlab.model}}

\begin{fulllineitems}
\phantomsection\label{\detokenize{modules/model:phlab.model.dist_disp_osc}}\pysiglinewithargsret{\sphinxbfcode{\sphinxupquote{class }}\sphinxcode{\sphinxupquote{phlab.model.}}\sphinxbfcode{\sphinxupquote{dist\_disp\_osc}}}{\emph{\DUrole{n}{inp\_dir}\DUrole{o}{=}\DUrole{default_value}{\textquotesingle{}./\_input/\textquotesingle{}}}, \emph{\DUrole{n}{out\_dir}\DUrole{o}{=}\DUrole{default_value}{\textquotesingle{}./\_output/\textquotesingle{}}}, \emph{\DUrole{n}{nmodel}\DUrole{o}{=}\DUrole{default_value}{0}}, \emph{\DUrole{n}{name}\DUrole{o}{=}\DUrole{default_value}{\textquotesingle{}\textquotesingle{}}}}{}
Creates object for distorted and displaced harmonic oscillator model.
\begin{description}
\item[{Args:}] \leavevmode\begin{description}
\item[{inp\_dir: str}] \leavevmode
name of the input directory.

\item[{out\_dir: str}] \leavevmode
name of the output directory.

\item[{nmodel: int}] \leavevmode
serial number of the model.

\item[{name: str}] \leavevmode
name of the model.

\end{description}

\item[{Attributes:}] \leavevmode\begin{description}
\item[{input\_default: dict}] \leavevmode
dictionary with default input parameters.

\item[{input: dict}] \leavevmode
dictionary with current input parameters.

\item[{npoints: int}] \leavevmode
number of points in the spectrum.

\item[{spec\_max: float}] \leavevmode
max limt of enrgy loss.

\item[{spec\_min: float}] \leavevmode
min limt of enrgy loss.

\item[{param2fit: object}] \leavevmode
parameters to fit.

\item[{nruns: int}] \leavevmode
number of runs.

\item[{color: str}] \leavevmode
color of the line.

\item[{input\_class: object}] \leavevmode
returns input\_handler for this model.

\item[{x: float}] \leavevmode
energy loss in eV for the phonon contribution.

\item[{y: float}] \leavevmode
rixs intensities (arb. units) for the phonon contribution.

\item[{y\_norm: float}] \leavevmode
normalized rixs intensities (arb. units) for the phonon contribution.

\end{description}

\end{description}

\end{fulllineitems}

\index{double\_osc (class in phlab.model)@\spxentry{double\_osc}\spxextra{class in phlab.model}}

\begin{fulllineitems}
\phantomsection\label{\detokenize{modules/model:phlab.model.double_osc}}\pysiglinewithargsret{\sphinxbfcode{\sphinxupquote{class }}\sphinxcode{\sphinxupquote{phlab.model.}}\sphinxbfcode{\sphinxupquote{double\_osc}}}{\emph{\DUrole{n}{inp\_dir}\DUrole{o}{=}\DUrole{default_value}{\textquotesingle{}./\_input/\textquotesingle{}}}, \emph{\DUrole{n}{out\_dir}\DUrole{o}{=}\DUrole{default_value}{\textquotesingle{}./\_output/\textquotesingle{}}}, \emph{\DUrole{n}{nmodel}\DUrole{o}{=}\DUrole{default_value}{0}}, \emph{\DUrole{n}{name}\DUrole{o}{=}\DUrole{default_value}{\textquotesingle{}\textquotesingle{}}}}{}
Creates object for 2D harmonic oscillator model.
\begin{description}
\item[{Args:}] \leavevmode\begin{description}
\item[{inp\_dir: str}] \leavevmode
name of the input directory.

\item[{out\_dir: str}] \leavevmode
name of the output directory.

\item[{nmodel: int}] \leavevmode
id number of the model.

\item[{name: str}] \leavevmode
name of the model.

\end{description}

\item[{Attributes:}] \leavevmode\begin{description}
\item[{input\_default: dict}] \leavevmode
dictionary with default input parameters.

\item[{input: dict}] \leavevmode
dictionary with current input parameters.

\item[{npoints: int}] \leavevmode
number of points in the spectrum.

\item[{spec\_max: float}] \leavevmode
max limt of enrgy loss.

\item[{spec\_min: float}] \leavevmode
min limt of enrgy loss.

\item[{param2fit: object}] \leavevmode
parameters to fit.

\item[{nruns: int}] \leavevmode
number of runs.

\item[{color: str}] \leavevmode
color of the line.

\item[{input\_class: object}] \leavevmode
returns input\_handler for this model.

\item[{x: float}] \leavevmode
energy loss in eV for the phonon contribution.

\item[{y: float}] \leavevmode
rixs intensities (arb. units) for the phonon contribution.

\item[{y\_norm: float}] \leavevmode
normalized rixs intensities (arb. units) for the phonon contribution.

\end{description}

\end{description}

\end{fulllineitems}

\index{input\_handler (class in phlab.model)@\spxentry{input\_handler}\spxextra{class in phlab.model}}

\begin{fulllineitems}
\phantomsection\label{\detokenize{modules/model:phlab.model.input_handler}}\pysiglinewithargsret{\sphinxbfcode{\sphinxupquote{class }}\sphinxcode{\sphinxupquote{phlab.model.}}\sphinxbfcode{\sphinxupquote{input\_handler}}}{\emph{\DUrole{n}{input\_default}\DUrole{o}{=}\DUrole{default_value}{\{\}}}, \emph{\DUrole{n}{inp\_dir}\DUrole{o}{=}\DUrole{default_value}{\textquotesingle{}./\_input/\textquotesingle{}}}, \emph{\DUrole{n}{nmodel}\DUrole{o}{=}\DUrole{default_value}{1}}, \emph{\DUrole{n}{inp\_name}\DUrole{o}{=}\DUrole{default_value}{\textquotesingle{}input\_model\_\{nm\}.json\textquotesingle{}}}, \emph{\DUrole{n}{model\_name}\DUrole{o}{=}\DUrole{default_value}{\textquotesingle{}1d\textquotesingle{}}}}{}
Contains methods to read and update input.
\begin{description}
\item[{Args:}] \leavevmode\begin{description}
\item[{input\_default: dict}] \leavevmode
dictionary with input parameters

\item[{inp\_dir: str}] \leavevmode
name of the input directory

\item[{nmodel: int}] \leavevmode
id number of the model

\item[{model\_name: str}] \leavevmode
name of the model

\end{description}

\item[{Attributes:}] \leavevmode\begin{description}
\item[{input: dict}] \leavevmode
dictionary with input parameters

\end{description}

\end{description}

\end{fulllineitems}

\index{parameters2fit (class in phlab.model)@\spxentry{parameters2fit}\spxextra{class in phlab.model}}

\begin{fulllineitems}
\phantomsection\label{\detokenize{modules/model:phlab.model.parameters2fit}}\pysigline{\sphinxbfcode{\sphinxupquote{class }}\sphinxcode{\sphinxupquote{phlab.model.}}\sphinxbfcode{\sphinxupquote{parameters2fit}}}
Defines paramters to fit.
\begin{description}
\item[{Attributes:}] \leavevmode\begin{description}
\item[{dict: dict}] \leavevmode
dictionary with parameters to fit

\end{description}

\end{description}

\end{fulllineitems}

\index{single\_osc (class in phlab.model)@\spxentry{single\_osc}\spxextra{class in phlab.model}}

\begin{fulllineitems}
\phantomsection\label{\detokenize{modules/model:phlab.model.single_osc}}\pysiglinewithargsret{\sphinxbfcode{\sphinxupquote{class }}\sphinxcode{\sphinxupquote{phlab.model.}}\sphinxbfcode{\sphinxupquote{single\_osc}}}{\emph{\DUrole{n}{inp\_dir}\DUrole{o}{=}\DUrole{default_value}{\textquotesingle{}./\_input/\textquotesingle{}}}, \emph{\DUrole{n}{out\_dir}\DUrole{o}{=}\DUrole{default_value}{\textquotesingle{}./\_output/\textquotesingle{}}}, \emph{\DUrole{n}{nmodel}\DUrole{o}{=}\DUrole{default_value}{0}}, \emph{\DUrole{n}{name}\DUrole{o}{=}\DUrole{default_value}{\textquotesingle{}\textquotesingle{}}}}{}
Creates object for 1D harmonic oscillator model.
\begin{description}
\item[{Args:}] \leavevmode\begin{description}
\item[{inp\_dir: str}] \leavevmode
name of the input directory.

\item[{out\_dir: str}] \leavevmode
name of the output directory.

\item[{nmodel: int}] \leavevmode
id number of the model.

\item[{name: str}] \leavevmode
name of the model.

\end{description}

\item[{Attributes:}] \leavevmode\begin{description}
\item[{input\_default: dict}] \leavevmode
dictionary with default input parameters.

\item[{input: dict}] \leavevmode
dictionary with current input parameters.

\item[{npoints: int}] \leavevmode
number of points in the spectrum.

\item[{spec\_max: float}] \leavevmode
max limt of enrgy loss.

\item[{spec\_min: float}] \leavevmode
min limt of enrgy loss.

\item[{param2fit: object}] \leavevmode
parameters to fit.

\item[{nruns: int}] \leavevmode
number of runs.

\item[{color: str}] \leavevmode
color of the line

\item[{input\_class: object}] \leavevmode
returns input\_handler for this model.

\item[{x: float}] \leavevmode
energy loss in eV for the phonon contribution.

\item[{y: float}] \leavevmode
rixs intensity (arb. units) for the phonon contribution.

\item[{y\_norm: float}] \leavevmode
normalized rixs intensity (arb. units) for the phonon contribution.

\end{description}

\end{description}

\end{fulllineitems}



\section{experiment}
\label{\detokenize{modules/experiment:module-phlab.experiment}}\label{\detokenize{modules/experiment:experiment}}\label{\detokenize{modules/experiment::doc}}\index{module@\spxentry{module}!phlab.experiment@\spxentry{phlab.experiment}}\index{phlab.experiment@\spxentry{phlab.experiment}!module@\spxentry{module}}\index{experiment (class in phlab.experiment)@\spxentry{experiment}\spxextra{class in phlab.experiment}}

\begin{fulllineitems}
\phantomsection\label{\detokenize{modules/experiment:phlab.experiment.experiment}}\pysiglinewithargsret{\sphinxbfcode{\sphinxupquote{class }}\sphinxcode{\sphinxupquote{phlab.experiment.}}\sphinxbfcode{\sphinxupquote{experiment}}}{\emph{\DUrole{n}{expfile}\DUrole{o}{=}\DUrole{default_value}{\textquotesingle{}\textquotesingle{}}}, \emph{\DUrole{n}{columns}\DUrole{o}{=}\DUrole{default_value}{{[}0, 1{]}}}, \emph{\DUrole{n}{nexp}\DUrole{o}{=}\DUrole{default_value}{1}}, \emph{\DUrole{n}{name}\DUrole{o}{=}\DUrole{default_value}{\textquotesingle{}\textquotesingle{}}}}{}
Experiment.
\begin{description}
\item[{Args:}] \leavevmode\begin{description}
\item[{file: str}] \leavevmode
path to the file with the exp data.

\item[{col: list}] \leavevmode
{[}column x ; column y{]} defines which columns to read from the file.

\item[{name: str}] \leavevmode
name of the experiment.

\item[{nexp: int}] \leavevmode
id number of the given experiment in the given project.

\end{description}

\item[{Attributes:}] \leavevmode\begin{description}
\item[{x: float}] \leavevmode
energy loss readings from exp file.

\item[{y: float}] \leavevmode
rixs intensity readings from  exp file.

\item[{max: float}] \leavevmode
max value of y.

\item[{y\_norm: float}] \leavevmode
normalized y.

\item[{name: str}] \leavevmode
name of the experiment.

\item[{xmin: float}] \leavevmode
min value of x.

\item[{xmax: float}] \leavevmode
max value of x.

\end{description}

\end{description}

\end{fulllineitems}



\section{visual}
\label{\detokenize{modules/visual:module-phlab.visual}}\label{\detokenize{modules/visual:visual}}\label{\detokenize{modules/visual::doc}}\index{module@\spxentry{module}!phlab.visual@\spxentry{phlab.visual}}\index{phlab.visual@\spxentry{phlab.visual}!module@\spxentry{module}}\index{plot (class in phlab.visual)@\spxentry{plot}\spxextra{class in phlab.visual}}

\begin{fulllineitems}
\phantomsection\label{\detokenize{modules/visual:phlab.visual.plot}}\pysiglinewithargsret{\sphinxbfcode{\sphinxupquote{class }}\sphinxcode{\sphinxupquote{phlab.visual.}}\sphinxbfcode{\sphinxupquote{plot}}}{\emph{\DUrole{n}{model\_list}\DUrole{o}{=}\DUrole{default_value}{{[}{]}}}, \emph{\DUrole{n}{exp}\DUrole{o}{=}\DUrole{default_value}{{[}{]}}}}{}
Visualization.
\begin{description}
\item[{Args:}] \leavevmode\begin{description}
\item[{model\_list: list}] \leavevmode
list of models.

\item[{exp: object}] \leavevmode
experiment.

\end{description}

\item[{Attributes:}] \leavevmode\begin{description}
\item[{if\_exp: boolen}] \leavevmode
returns True if experiment (object) is specified.

\end{description}

\end{description}

\end{fulllineitems}



\renewcommand{\indexname}{Python Module Index}
\begin{sphinxtheindex}
\let\bigletter\sphinxstyleindexlettergroup
\bigletter{p}
\item\relax\sphinxstyleindexentry{phlab}\sphinxstyleindexpageref{modules/main:\detokenize{module-phlab}}
\item\relax\sphinxstyleindexentry{phlab.experiment}\sphinxstyleindexpageref{modules/experiment:\detokenize{module-phlab.experiment}}
\item\relax\sphinxstyleindexentry{phlab.model}\sphinxstyleindexpageref{modules/model:\detokenize{module-phlab.model}}
\item\relax\sphinxstyleindexentry{phlab.visual}\sphinxstyleindexpageref{modules/visual:\detokenize{module-phlab.visual}}
\end{sphinxtheindex}

\renewcommand{\indexname}{Index}
\printindex
\end{document}